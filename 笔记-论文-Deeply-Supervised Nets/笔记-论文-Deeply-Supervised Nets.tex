\documentclass{article}
\usepackage{CJK}
\usepackage{color}
\usepackage{amsmath}
\usepackage{geometry}
\geometry{left=0.5cm,right=0.5cm,top=0.5cm,bottom=0.5cm}
\begin{document}
\begin{CJK}{UTF8}{gbsn}
\LARGE
%\title{笔记-论文-Deeply-Supervised Nets}
%\author{甘宇飞}
%\date{\today}
%\maketitle
\section{Introduction}
此文在CNN的基础上对hidden layer进行supervision。实际上是对CNN的隐层采用squared hinge loss (L2 Loss)使得CNN的隐层具有discriminative。
本文的方法有一个先验的经验,那就是如果能够让特征discriminative,则分类效果一定就好。(不过这一点我觉得不能完全认同)

Introduction部分指出了现在CNN存在的一些问题,觉得有必要列出来:
\begin{itemize}
\item 隐层的特征的 transparency 和 criminativeness 特性减少。
\item 由 exploding and vanishing gradients 导致的训练难度
\item 缺乏对DL算法的数学理论解释
\item 对大量训练数据的依赖
\item 调参复杂
\end{itemize}

\section{算法说明}
\subsection{符号规定}
input training data: 
$S=\{(X_i,y_i),i=1,\dots,N\}$ 其中$X_i\in\mathcal{R}^n$表示原始数据,$y_i\in\{1,\dots,K\}$表示对应的label。\\
网络层数:$M$ \\
学习到的filters/weights:$W^{(m)},m=1,\dots\,M$ \\
$m-1$层产生的feature map:$Z^{(m-1)}$ \\
convolved/filtered responses:$Q^{(m)}$ \\
Pooling function:$f()$ \\
输出层的SVM weights:$\mathrm{w}^{(out)}$
\subsection{公式说明}
对每一层$m=1,\dots\,M$有:
\begin{equation} Z^{(m)}=f(Q^{(m)}),~and~Z^{(0)}\equiv X, \end{equation}
\begin{equation} Q^{(m)}=W^{(m)}\ast Z^{(m-1)}, \end{equation}
亦即从网络的结构上来看文中的结构与传统的CNN网络并无差别。
合并所有的层的weights有:
$$W=(W^{(1)},\dots,W^{(M)}),$$
对于each hidden layer有:
$$\mathrm{w}=(\mathrm{w}^{(1)},\dots,\mathrm{w}^{(M-1)}),$$
{\color{red}\%\%}这个w是怎么来的我目前还没完全确定,文章写的不是十分清楚,智能说一下自己的理解,这个w很可能是将$Z^{(m-1)}$作为输入,输出为各不相同的label到L2SVM中,也就是说利用svm将一层的$Z$完全区分开来,这样学到一个对应的w。因此在训练之前这个w是未知的,他与每一次filters update时一起update,其update方法与CNN+SVM的策略相同,在《Deep Learning using Linear Support Vector Machines》中有详细介绍(本文后面附有链接)。

{\color{red}本文的核心motivation就是在hidden layer加入约束}


总体上的目标函数是:
\begin{equation}\label{eq:3}
 \|\mathrm{w}^{(out)}\|^2+\mathcal{L}(W,\mathrm{w}^{(out)})+\sum^{M-1}_{m=1}\alpha_m\left[\|\mathrm{w}^{(out)}\|^2+\ell(W,\mathrm{w}^{(m)})-\gamma\right]_+, \end{equation}
这其中:
\begin{equation}
\mathcal{L}(W,\mathrm{w}^{(out)})=\sum_{y_k\neq y}\left[1-\langle\mathrm{w}^{(out)},\phi(Z^{(M)},y)-\phi(Z^{(M)},y_k)\rangle\right]^2_+
\end{equation}
\begin{equation} 
\ell(W,\mathrm{w}^{(m)})=\sum_{y_k\neq y}\left[1-\langle\mathrm{w}^{(m)},\phi(Z^{(m)},y)-\phi(Z^{(m)},y_k)\rangle\right]^2_+
\end{equation}
对于这个目标函数有如下解释:
\begin{itemize}
\item\eqref{eq:3}左边两项$\|\mathrm{w}^{(out)}\|^2$和$\mathcal{L}(W,\mathrm{w}^{(out)})$与传统CNN相似,不同的是在这里不是用传统CNN的softmax做分类器而是用SVM。
\item\eqref{eq:3}右边一项括号中的$\|\mathrm{w}^{(out)}\|^2$和$\ell(W,\mathrm{w}^{(m)})$是本文中提出的方法
\item从整体上来看这个函数,一方面它照顾到了经典CNN中误差反向传导所必须的自下至上的传递结构,另一方面还使得filters之间满足response discriminative的特性。
\end{itemize}
未完待续。。。。

\section{题外话}
这篇文章值得注意的地方不仅仅是方法,更重要的是思路,虽然很多人认为中层监督相比普通CNN相对退化,也就是说现在主流研究还应该是用理论解释how does DL work?而不是有退回了原来的特征+聚类+分类的思路。但是我认为中层监督实际上还是存在实际应用上的价值,尤其是语意上的中层监督,可以将人工的引导引入CNN的中层,这样好处有3:(1)大量label大有用武之地(2)帮助解释中层black box(3)直觉上能够提高收敛速度。





\end{CJK}
\end{document}