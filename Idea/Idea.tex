\documentclass{article}
\usepackage{CJK}
\usepackage{color}
\usepackage{amsmath}
\usepackage{geometry}
\usepackage{booktabs}
%\usepackage{hyperref}
%\geometry{left=0.5cm,right=0.5cm,top=0.5cm,bottom=0.5cm}
\begin{document}
\begin{CJK}{UTF8}{gbsn}
%\LARGE
\title{Idea}
%\author{甘宇飞}
%\date{\today}
%\maketitle

{
\color{blue}
\section*{约定}
\subsection{在DL中靠近输入图像的一层或多层称之为底层}
\subsection{靠近输出层的一层或多层为顶层/分类层}
\subsection{相对地既不靠近输入层又不靠近输出层的一层或多层称为中间层}
}

\section{深度网络研究现状(for classification)}
\subsection{面向底层}
加强最底层的特征的表征能力\\
\textbf{代表:}各种pre-training方法,auto-encoding,Multi-view\cite{sermanet2013overfeat}等等。
\subsection{面向顶层}
加强CNN的分类能力\\
\textbf{代表:}softmax代替Logistic Regression\cite{karpathy2014large},后又有人用SVM代替softmax\cite{tang2013deep}\cite{nagi2012convolutional}
\subsection{面向中间层}
加强中间层的高级语意特征,多为卷积层干预。中间层干预在GoogLeNet中已有人提出到DSN时得到理论验证\\
\textbf{代表:}Network in Network\cite{NIN},Maxout networks\cite{goodfellow2013maxout},与Part based model组合\cite{ouyang2013joint}(有篇文章甚至说PBM就是CNN\cite{zeiler2011adaptive}),GoogLeNet\cite{szegedy2014going},DSN\cite{lee2014deeply},以及对中层图像的重构分析\cite{zeiler2014visualizing}
\subsection{面向Pooling}
这一点在以前的笔记中间已经提到,MAXPooling会丢掉信息,最近Hinton也已经提到这个问题2014.11.11 http://www.reddit.com/r/MachineLearning/comments/2lmo0l/ama\_geoffrey\_hinton/\\
\textbf{代表:}OverFeat\cite{sermanet2013overfeat}
\subsection{面向BP}
加强特征的discriminative特性,这部分实际上有不少工作是与上面相互交叉的。\\
\textbf{代表:}对中间层加强discriminative\cite{lee2014deeply}
\subsection{多层结构}
只利用多层结构不考虑降维,但保留了卷积操作。\\
\textbf{代表:}PCANet\cite{chan2014pcanet}
\section{可供改进的方向}
\subsection{自学习结构(by He)}
多重结构相互组合从其中选择效果好的结构。\\
\subsubsection{难点:}
1、各个模块之间的特征传递形式不同(例如统计模型的参数不能像CNN那样做卷积)\\
2、判断孰优孰劣的条件(如果要训练网络用何种方法调整网络参数?)\\
\subsubsection{方案:}
%Deep learning一般误差是由最后一层反向传播到前面的层,也就是说只考虑到的最后的误差。尝试在中间层加上loss项以提高效果。\\
多层的网络结构优点是可以更好地学习特征(但目前仍然缺乏理论解释为什么多层特征学习效果会好),合理利用这种结构特点或许能够设计效果更好的分类网络。\\
基本的分类方法概括为求基底$\rightarrow$特征映射、编码$\rightarrow$分类。而寻找基底和编码这些步骤是可以有许多替换的。例如基底可以由CNN,PCA,LDA,CS等方法学习得到,而特征映射以及编码又可以由BoW,卷积+MAXPooling,还有大量的的降维方法可以运用于这一部分。对于分类则有LR,SVM,SoftMax。\\
对与\textbf{难点2}的解决方法:中间层如果以discriminative作为loss function(这是一种靠直觉的的干预,特征越discriminative图像分类效果越好)。\\
%自上而下的误差传导,“实践是检验真理的唯一标准”,有更好的效果但是无法保证能够收敛到全局最优。\\
\subsubsection{公式:}
抽象描述:\\
\begin{equation}\label{eq:1}
loss=\sum^{M-1}_{m=1}\alpha_m\left[\|W^{(m)}\|^2+\ell(W^{(m)})\right]
\end{equation}
$\ell()$代表各层产生的loss,$\ell()$的设计反映我们需要考虑的特征属性,\\
详细描述:\\
1、模型采用改进的反向误差传导(Loss不仅仅由顶层产生,还又中间层以及底层的响应/特征映射产生),这里可以加入一些诸如discriminative的特征属性用来约束特征的学习。这里关于$L_1$与$L_2$的相关文章探索他们的属性\cite{lee2013study}\cite{moore2011l1}\\
策略有很多例如:\\
discriminative: hinge-squared-loss\cite{lee2014deeply}
\begin{equation}\label{eq:2}
 \|\mathrm{w}^{(out)}\|^2+\mathcal{L}(W,\mathrm{w}^{(out)})+\sum^{M-1}_{m=1}\alpha_m\left[\|\mathrm{w}^{(out)}\|^2+\ell(W,\mathrm{w}^{(m)})-\gamma\right]_+, \end{equation}
这其中:
\begin{equation}
\mathcal{L}(W,\mathrm{w}^{(out)})=\sum_{y_k\neq y}\left[1-\langle\mathrm{w}^{(out)},\phi(Z^{(M)},y)-\phi(Z^{(M)},y_k)\rangle\right]^2_+
\end{equation}
\begin{equation} 
\ell(W,\mathrm{w}^{(m)})=\sum_{y_k\neq y}\left[1-\langle\mathrm{w}^{(m)},\phi(Z^{(m)},y)-\phi(Z^{(m)},y_k)\rangle\right]^2_+
\end{equation}
以及k-nearest neighbor classier等等\\
2、由于暂时无法解决参数传递问题,目前拟依然以卷积操作最为求特征映射的操作。(因此对于一部分用于降维的模型必须恢复出降维前的的向量),亦即这种形式的操作予以保留\\
\begin{equation} Z^{(m)}=f(Q^{(m)}), \end{equation}
\begin{equation} Q^{(m)}=W^{(m)}\ast Z^{(m-1)}, \end{equation}
3、优化方法和数据集与主流方法变化不大。\\
梯度下降,利用\eqref{eq:1}求$\frac{\partial{loss}}{\partial{W}}$,相关理论工作\cite{shamir2012stochastic},SGD的有点有可以在线学习以及收敛速度快。
\subsubsection{实验}
\begin{table}[!ht]\
\caption{cifar-10}
~\\
\centering
\begin{tabular}{rl} \toprule 
Result	&   Method \\ \midrule
94.00\%		&   Lessons learned from manually classifying CIFAR-10\\  \midrule
91.78\%	&   Deeply-Supervised Nets\\  \midrule
91.20\%	&   Network In Network\\  \midrule
90.68\%	&   Regularization of Neural Networks using DropConnect\\  \midrule
90.65\%	&   Maxout Networks\\   \midrule
\textbf{64.10\%}    &   \textbf{me}\\ \bottomrule
\end{tabular}\\
\end{table}

\begin{table}[!ht]\
\caption{cifar-100}
~\\
\centering
\begin{tabular}{rl} \toprule 
Result	&   Method \\ \midrule
65.43\%	&   Deeply-Supervised Nets\\  \midrule
64.32\%	&   Network In Network\\  \midrule
63.15\%	&   Discriminative Transfer Learning with Tree-based Priors\\  \midrule
61.86\%	&   Improving Deep Neural Networks with Probabilistic Maxout Units\\  \midrule
61.43\%	&   Maxout Networks\\ \bottomrule
\end{tabular}\\
\end{table}

似乎有较快的收敛速度。还需要进一步研究$\alpha_m$的取值方法。

\bibliographystyle{plain}
\bibliography{G}
\end{CJK}
\end{document}